\section{表与图}这节用来展示表格与图片的插入。

\subsection{插图}
图\ref{figure1}是通栏图的演示:
\begin{figure}[hbtp]
    \centering
    \includegraphics[width=100mm]{example-image}
    \caption{图片测试(最小宽度)\\Image test (Minimal width)}
    \label{figure1}
\end{figure}
\par 注意:这里为了减少图片上下的空白,使用了float宏包。

% 图\ref{subfigures}是一个多栏图的演示:

% \begin{figure*}[h]\center
%     \begin{subfigure}[t]{0.25\textwidth}
%         \centering
%         \includegraphics[width=\textwidth]{example-image}
%         \caption{subfigure1}
%         \label{subfigure1}
%     \end{subfigure}
%     \begin{subfigure}[t]{0.25\textwidth}
%         \centering
%         \includegraphics[width=\textwidth]{example-image}
%         \caption{subfigure2}
%         \label{subfigure2}
%     \end{subfigure}
%     \begin{subfigure}[t]{0.25\textwidth}
%         \centering
%         \includegraphics[width=\textwidth]{example-image}
%         \caption{subfigure3}
%         \label{subfigure3}
%     \end{subfigure}
%     \begin{subfigure}[t]{0.25\textwidth}
%         \centering
%         \includegraphics[width=\textwidth]{example-image}
%         \caption{subfigure4}
%         \label{subfigure4}
%     \end{subfigure}
%     \begin{subfigure}[t]{0.25\textwidth}
%         \centering
%         \includegraphics[width=\textwidth]{example-image}
%         \caption{subfigure5}
%         \label{subfigure5}
%     \end{subfigure}
%         \begin{subfigure}[t]{0.25\textwidth}
%         \centering
%         \includegraphics[width=\textwidth]{example-image}
%         \caption{subfigure6}
%         \label{subfigure6}
%     \end{subfigure}
%     \caption{多栏图示例}
%     \label{subfigures}
% \end{figure*}

\subsection{表格}
\par 本来LaTeX里表格的变化是非常多的,但论文中常用三线式,问题反而简单了。下表\ref{tableExample}是一个例子:
\begin{table}[htbp]\center
    \caption{示例表格\\Example Table}
    \begin{tabular}{lcccccl}
        \toprule
        。。 & 。。 & 。。 & 。。 & 。。& 。。 & 。。\\
        \midrule
        。。 & 。。 & 。。 & 。。 & 。。& 。。 & 。。\\
        。。 & 。。 & 。。 & 。。 & 。。& 。。 & 。。\\
        。。 & 。。 & 。。 & 。。 & 。。& 。。 & 。。\\
        。。 & 。。 & 。。 & 。。 & 。。& 。。 & 。。\\
        。。 & 。。 & 。。 & 。。 & 。。& 。。 & 。。\\
        \bottomrule
    \end{tabular}
    \label{tableExample}
\end{table}
如果你有使用更复杂的表格的需求,请自行查资料完成。