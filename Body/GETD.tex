\section{共轭梯度法}
\subsection{线性共轭梯度法}
    一般的共轭梯度法由线性共轭梯度法推导而来,
    所以我们首先考虑关于正定二次函数的共轭梯度法,先有以下定义:
    
    设G是n阶对称正定矩阵。若$R^n$中两个非零向量$d_i,d_j(i\not=j)$满足
    \begin{equation}
        d_i^TGd_j=0,\quad\quad i,j=0,...,n-1,i\not=j,
        \nonumber
    \end{equation}则称$d_i$和$d_j$是共轭方向。

    若$R^n$中n个非零向量$d_0,...,dd_{n-1}$满足
    \begin{equation}
        d_i^TGd_j=0,\quad\quad i,j=0,...,n-1,i\not=j,
        \nonumber
    \end{equation}
    则称它们为G的两两共轭方向,或称这个向量组是共轭的。

    共轭方向是根据正定二次函数$f(x)=\frac{1}{2}x^TGx+b^Tx$的梯度来构造的,
    所以我们可以由此导出共轭梯度法的迭代方向$d_k=-g_k+\beta_{k-1}d_{k-1}$。
    因为对二次正定函数有$g_k^Tg_{k-1}=0$,
    可以得到$\beta_{k-1}=\displaystyle\frac{g_k^Tg_k}{g_{k-1}^Tg_{k-1}}$。
    \begin{algorithm}
        \SetKwInOut{Input}{输入}\SetKwInOut{Output}{输出}
    
        \SetAlgoLined
        \Input{$x_0\in \mathrm {R}^{n}, d_0=-g_0, \varepsilon>0$,初始化设置$k=0$}
        \Output{最优解$x_{k+1}$}
    
        \While{未收敛} {
            $\alpha_k=-\displaystyle\frac{g_k^Td_k}{d_k^TGd_k}$
            
            $x_{k+1}=x_k+\alpha_kd_k$
            
            $d_{k+1}=-g_{k+1}+\beta_kd_k,k=k+1$
        }
        \caption{线性共轭梯度法的算法}
    \end{algorithm}
                     
\subsection{非线性共轭梯度法}
由线性共轭梯度法推广而来,迭代方向与线性共轭梯度相同,
由以下两个不同的非线性共轭梯度公式,
可以给出非线性共轭梯度方法的算法步骤。

FR公式:
\begin{equation}
    \beta_{k-1}^{FR}=\displaystyle\frac{g_k^Tg_k}{g_{k-1}^Tg_{k-1}} .
    \nonumber
\end{equation}
        
PRP公式:
\begin{equation}
    \beta_{k-1}^{PRP}=\displaystyle\frac{g_k^T(g_k-g_{k-1})}{g_{k-1}^Tg_{k-1}} .
    \nonumber
\end{equation}

\begin{algorithm}
    \SetKwInOut{Input}{输入}\SetKwInOut{Output}{输出}

    \SetAlgoLined
    \Input{$x_0\in \mathrm {R}^{n}, d_0=-g_0, \varepsilon>0$,初始化设置$k=0$}
    \Output{最优解$x_{k+1}$}

    \While{未收敛} {
        一维线搜索求$\alpha_k$
        
        $x_{k+1}=x_k+\alpha_kd_k$

        $\beta_k,d_{k+1}=-g_{k+1}+\beta+kd_k,k=k+1$
    }
    \caption{非线性共轭梯度法的算法}
\end{algorithm}

\subsubsection{非线性共轭梯度法FR的算例}
\begin{example}
    \begin{equation}
        \min f(x) = \frac{3}{2}x_1^2+\frac{1}{2}x_2^2-x_1x_2-2x_1,
        x_0 = (0,0)^T.
    \nonumber
    \end{equation}
\end{example}
\begin{solution}
    \begin{equation}
        g(x) = 
        \begin{pmatrix}
            3x_1-x_2-2 \\
            -x_1+x_2  
        \end{pmatrix},
        G(x) = 
        \begin{pmatrix}
            3 & -1\\
            -1 & 1   
        \end{pmatrix}.
    \nonumber    
    \end{equation}
    因$g_0=(-2,0)^T$,故取$d_0=(2,0)^T$,
    从$x_0$出发,沿$d_0$作一维搜索,即求
    \begin{equation*}
    \min f(x_0+\alpha_0 d_0) = 6\alpha^2-4\alpha ,
    \end{equation*}
    的极小点,得步长$\alpha_0 = \frac{1}{3}$。于是得到
    \begin{equation*}
        x_1=x_0 + \alpha_0d_0=(\frac{2}{3},0)^T,
        g_1 = (0,-\frac{2}{3}).
    \end{equation*}
    由FR公式得
    \begin{equation*}
        \beta_{0}^{FR}=\displaystyle\frac{g_1^Tg_1}{g_{0}^Tg_{0}} = \frac{1}{9},
    \end{equation*}
    故
    \begin{equation*}
        d_1 = -g_1 + \beta_0d_0=(\frac{2}{9},\frac{2}{3})^T.
    \end{equation*}
    从$x_1$出发,沿$d_1$作一维搜索,求
    \begin{equation}
        \min f(x_1+\alpha_1 d_1) = 
        \frac{4}{27}\alpha^2-\frac{4}{9}\alpha + \frac{2}{3} ,
        \nonumber
    \end{equation}
    的极小点,解得$\alpha_1= \frac{3}{2}$,
    于是$x_2 = x_1 + \alpha_1d_1 = (1,1)^T$。
    此时$d_2=(0,0)^T$,故$x_{k+1}=(1,1)^T$,$f_{k+1} = -1$。
\end{solution}

\subsubsection{非线性共轭梯度法的理论结果}
\begin{theorem}
    对于FR方法,若$\alpha_k$由强Wolfe准则得到,
    且$\sigma\in(0,\frac{1}{2})$,则$d_k$满足
    \begin{equation}
        -\displaystyle\sum^{k-1}_{j=0}\sigma^j \leq
        \displaystyle\frac{g_k^Td_k}{\|g_k\|^2}\leq
        -2 + \displaystyle\sum^{k-1}_{j=0}\sigma^j, k = 1...,
        \label{getd31}
    \end{equation}
    其中$d_k$是下降方向\cite{1992Global}。
    \label{71}
\end{theorem}
\begin{proof}   
    通过数学归纳法进行证明。
    由于式\ref{getd31}这个结果显然适用于k=1,
    因为三项均为-1。
    假设式\ref{getd31}对某些k>1成立,
    这意味着$<gk,dk><0$,因为
    \begin{equation*}
        -2 + \displaystyle\sum^{k-1}_{j=0}\sigma^J < 
        -2 + \frac{1}{1-\sigma} = 
        \frac{2\sigma-1}{1-\sigma} < 0,
    \end{equation*}
    在$\sigma < \frac{1}{2}$时,根据FR公式,我们得到
    \begin{equation*}\label{getd32}
        \displaystyle\frac{g_{k+1}^Td_{k+1}}{\|g_{k+1}\|^2} = 
        -1 + \beta_{k+1}\displaystyle\frac{g_{k+1}^Td_k}{\|g_{k+1}\|^2} = 
        -1 + \displaystyle\frac{\beta_{k+1}}{\beta_{k+1}^{\mathrm {FR}}}
        \displaystyle\frac{g_{k+1}^Td_k}{\|g_{k}\|^2}.
    \end{equation*}
    根据线搜索条件,我们得到
    \begin{equation*}
        |\beta+{k+1}(g_{k+1}^Td_k)| \leq -\sigma||\beta+{k+1}|(g_{k}^Td_k),
    \end{equation*}
    将上式与式\ref{getd32}联立,得到
    \begin{equation}
        -1 + \sigma \displaystyle\frac{|\beta_{k+1}|}{\beta_{k+1}^{\mathrm {FR}}}
        \displaystyle\frac{g_{k}^Td_k}{\|g_{k}\|^2}
        \leq
        \displaystyle\frac{g_{k+1}^Td_{k+1}}{\|g_{k}\|^2}
        \leq
        -1 - \sigma \displaystyle\frac{|\beta_{k+1}|}{\beta_{k+1}^{\mathrm {FR}}}
        \displaystyle\frac{g_{k}^Td_k}{\|g_{k}\|^2}.
    \nonumber
    \end{equation}
    引入定理\ref{71}中的左侧假设条件,我们可以得到
    \begin{equation}
        -1 - \sigma \displaystyle\frac{|\beta_{k+1}|}{\beta_{k+1}^{\mathrm {FR}}}
        \displaystyle\sum^{k-1}_{j=0}\sigma^j
        \leq
        \displaystyle\frac{g_{k+1}^Td_{k+1}}{\|g_{k}\|^2}
        \leq
        -1 + \sigma \displaystyle\frac{|\beta_{k+1}|}{\beta_{k+1}^{\mathrm {FR}}}
        \displaystyle\sum^{k-1}_{j=0}\sigma^j.
    \nonumber
    \end{equation}
    由于$|\beta_k| \leq \beta_k^{\mathrm {FR}}$,
    并且$\displaystyle\sum^{k-1}_{j=0}\sigma^{j+1} = \displaystyle\sum^{k}_{j=0}\sigma^{j} - 1 $,
    我们可以得到定理\ref{71}对于$k+1$也成立,故数学归纳法成立。
\end{proof}

    如果采用非精确线搜索,FR和PRP方法都有可能产生式上升的方向。
    而该定理告诉我们对于FR方法而言,只有当使用强Wolfe线搜索,
    并且保证$\sigma\in(0,\frac{1}{2})$时,得到的方向才是下降方向。
    
\begin{theorem}[使用精确线搜索的FR方法的收敛性]\cite{1992Global}
    设有$f(x)$下界,$g(x)$满足Lipschitz条件,
    对使用精确线搜索准则的FR方法,
    则存在N,使得$g_N=0$,或者$\displaystyle\lim_{k\rightarrow\infty} \inf \|g_k\|=0$。
\end{theorem}
\begin{proof}
    根据定理\ref{71}以及在强Wolfe准则下的FR公式$|g(x_k+\alpha_kd_k)^Td_k| \leq -\sigma g_k^Td_k$,
    我们得到
    \begin{equation*}
        \begin{aligned}
        |g_k^Td_{k-1}|  &\leq -\sigma g_{k-1}^Td_{k-1}
                        &\leq \sigma \displaystyle\sum^{k-2}_{j=0}\sigma^j\|g_{k-1}\|^2
                        &\leq \displaystyle\frac{\sigma}{1-\sigma}\sigma^j\|g_{k-1}\|^2.
        \end{aligned}
    \end{equation*}
    根据梯度下降方向以及定理\ref{71},可以得到
    \begin{equation*}
        \begin{aligned}
        \|d_k\|^2       &\leq \|g_k\|^2 + 2|\beta_k| g_{k}^Td_{k-1} + \beta_k^2\|d_{k-1}\|^2
                        &\leq \|g_k\|^2 + \displaystyle\frac{2\sigma}{1-\sigma}|\beta_k| \|g_{k-1}\|^2 + \beta_k^2\|d_{k-1}\|^2
                        &\leq \displaystyle\frac{1+\sigma}{1-\sigma}\|g_k\|^2 + \beta_k^2\|d_{k-1}\|^2.
        \end{aligned}
    \end{equation*}
    重复使用这一关系,
    我们定义$\hat{\sigma} := \displaystyle\frac{1+\sigma}{1-\sigma} \geq 1$,
    并且使用条件$|\beta_k| \leq \beta_k^{\mathrm {FR}}$,
    得到
    \begin{equation*}
        \begin{aligned}
        \|d_k\|^2       &\leq \hat{\sigma}\|g_k\|^2 + \beta_k^2(\hat{\sigma}\|g_{k-1}\|^2 + \beta_{k-1}^2
                                                                                            \|d_{k-2}\|^2)
                        &\leq \hat{\sigma}\|g_k\|^4 + \displaystyle\sum_{j=1}^k\|g_j\|^{-2}.
        \end{aligned}
    \end{equation*}
    我们假设对于所有的$k$,都有$\|g_k\| \geq \gamma > 0$,这表示
    \begin{equation*}
        \|d_k\|^2 \leq \displaystyle\frac{\hat{\sigma}\gamma^4}{\gamma^2}k.
    \end{equation*}
    由此得证
\end{proof}
    该定理从理论上证明了精确线搜索下FR方法的收敛性,通过定理\ref{71},可以直到使用强Wolfe线搜索的FR方法在$\sigma\in(0,\frac{1}{2})$时有同样的收敛结果。而证明这一定理我们使用了Zoutendijk条件,其使用率相对广泛。

\begin{theorem}[Zoutendijk条件]\cite{Book1}
    设有$f(x)$下界,$g(x)$满足Lipschitz条件,
    使用Wolfe线搜索或精确线搜索准则的,
    具有$x_{k+1}=x_{k}+\alpha_kd_k$迭代格式的
    一般下降方法满足Zoutendijk条件:
    \begin{equation*}
    \displaystyle\sum_{k\geq0}\frac{(g_k^Td_k)^2}{\|d_k\|^2}<\infty .
    \end{equation*}
\end{theorem}
\begin{proof}
    设存在常数$\eta$,对任意$k\geq0$,有.利用Zoutendijk条件
    \begin{equation}
        \displaystyle\sum_{k\geq0}\frac{(g_k^\mathrm {T}d_k)^2}{\|d_k\|^2}<\infty=
        \displaystyle\sum_{k\geq0}\|g_k\|^2\cos^2\theta_k < \infty,
        \nonumber
    \end{equation}
    其中$\theta_k = <d_k, -g_k>$,知$\displaystyle\lim_{k\to\infty}\cos\theta_k=0$。
    
    考虑到FR方法的迭代格式和精确线搜索的特点,
    在第$k$次迭代,我们有$\|d_k\|=\|g_k\|\sec\theta_k $;
    在第$k+1$次迭代,我们有$\beta_k\|d_k\| = \|g_{k+1}\|\tan\theta_{k+1}$.
    结合这两个式子以及$\beta_k$的表示,就有
    \begin{equation}
        \tan\theta_{k+1}=\sec\theta_k\displaystyle\frac{\|g_{k+1}\|}{\|g_k\|},
        \nonumber
    \end{equation}
    上式两边平方,并利用$\sec^2\theta_k=1+\tan^2\theta_k$,便得到
    \begin{equation}
        \displaystyle\frac{\tan^2\theta_{k+1}}{\|g_{k+1}\|^2} = 
        \displaystyle\frac{1}{\|g_k\|^2} + 
        \displaystyle\frac{\tan^2\theta_k}{\|g_k\|^2}.
        \nonumber
    \end{equation}
    利用这个关系递推,可以得到
    \begin{equation}
        \displaystyle\frac{\tan^2\theta_k}{\|g_k\|^2} = 
        \displaystyle\sum_{i=0}^{k-1}\frac{1}{\|g_i\|^2},
        \nonumber
    \end{equation}
    其中$d_0=-g_0$,从而
    \begin{equation}
        \displaystyle\frac{\tan^2\theta_k}{\|g_k\|^2} \leq
        \displaystyle\frac{k}{\eta^2},
        \nonumber
    \end{equation}
    即
    \begin{equation}
        \displaystyle\frac{\eta^2}{k} \leq
        \displaystyle\frac{\|g_k\|^2}{\tan^2\theta_k} =
        \|g_k\|^2\cot^2\theta_k.
        \nonumber
    \end{equation}
    当充分大时,因为$\cot^2\theta_k\leq2\cos^2\theta_k$,就有
    \begin{equation}
        \displaystyle\sum_{k\geq0}\|g_k\|^2\cot^2\theta_k = \infty.
        \nonumber
    \end{equation}
    这与Zoutendijk条件矛盾。
\end{proof}