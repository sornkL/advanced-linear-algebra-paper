\section{引言}
凸优化问题是在凸集$X$上求解目标凸函数$f(\cdot)$最小值的一类最优化问题。
一般这类问题可以用于求解如何分配资源使收益达到最优,在对问题进行定性和定量的分析后,将资源与收益的关系抽象成适当的数学模型,作为目标函数$f(\cdot)$,根据模型的不同设计不同的优化方法进行求解。
也就是对于不同的目标函数$f(\cdot)$,可以使用不同的优化方法求解。

凸优化问题的求解在许多科学与工程领域有广泛的应用,例如在数据分析与机器学习、金融与经济、工业生产等方面都有具体应用。
对于机器学习和深度学习领域,在计算机视觉方向,早期AlexNet\cite{2012AlexNet}采用随机梯度下降法作为优化神经网络目标损失函数的算法;
后期,随着Transformer\cite{2017Transformer}在自然语言处理和计算机视觉方向的广泛应用,Adam\cite{2014Adam}成为了普遍采用的优化算法,包括Bert\cite{2018Bert}在内的语言模型也都采用了这一优化算法。

根据凸优化问题是否有约束,或根据目标函数的形式不同,分别有牛顿类方法与梯度类方法求解无约束优化问题,特别地,对于目标函数存在不可微的情况,可以在梯度概念的基础上,引入临近梯度,适用临近梯度下降法求解;
而对于有约束的优化问题,可以通过引入乘子,适用乘子法进行求解。本文分析上述经典优化方法,并结合实例求解具体凸优化问题。